\documentclass[a4paper]{article}
\usepackage[utf8]{inputenc}
\usepackage[a4paper, margin=1in]{geometry}
\usepackage{enumitem}
\usepackage{hyperref}
\usepackage{listings}
\usepackage{tabularx}
\usepackage{xcolor}
\usepackage{amsmath}
\usepackage{mathtools}
\DeclarePairedDelimiter\ceil{\lceil}{\rceil}
\DeclarePairedDelimiter\floor{\lfloor}{\rfloor}

\renewcommand{\arraystretch}{1.5}

\definecolor{codegreen}{rgb}{0,0.6,0}
\definecolor{codegray}{rgb}{0.5,0.5,0.5}
\definecolor{codepurple}{rgb}{0.58,0,0.82}
\definecolor{backcolour}{rgb}{0.95,0.95,0.92}

\lstdefinestyle{mystyle}{
    backgroundcolor=\color{backcolour},
    commentstyle=\color{codegreen},
    keywordstyle=\color{magenta},
    numberstyle=\tiny\color{codegray},
    stringstyle=\color{codepurple},
    basicstyle=\ttfamily\footnotesize,
    breakatwhitespace=false,
    breaklines=true,
    captionpos=b,
    keepspaces=true,
    numbers=left,
    numbersep=5pt,
    showspaces=false,
    showstringspaces=false,
    showtabs=false,
    tabsize=2
}
\lstset{style=mystyle}

\title{Computer Architecture HW5}
\author{Fabian Wüthrich}

\begin{document}

\maketitle

\section{Critical Paper Reviews [1000 points]}

see \href{https://safari.ethz.ch/review/architecture20/}{here}

\section{Parallel Speedup [200 points]}

\begin{enumerate}[label=\alph*)]
    \item Data sharing between cores could cause cache ping ponging

    \item We can calculate the speedup using Amdahl's law
        \[ \frac{1}{0.1 + \frac{0.9}{n}}  \xrightarrow{N\to\infty} \frac{1}{0.1} = 10\mathrm{x} \]

    \item Using Amdahl's law we can solve the following equation for $n$
        \[ 4 = \frac{1}{0.1 + \frac{0.9}{n}} \]
        Thus, we need 6 processors to achieve a speedup of 4.
    \item
        \begin{enumerate}[label=\roman*)]
            \item A large core of size $n^2$ yields a speedup of $n$ for the
            serial portion of the code and $16 - n^2$ small cores are available
                to execute the parallel portion. Using Amdahl's law we get the
                following speedup
                \[ \frac{1}{\frac{0.1}{n} + \frac{0.9}{16-n^2}} \]
                To maximize the speedup, we need to minimize the denominator.

                $n=1$ denominator 0.16

                $n=2$ denominator 0.125

                $n=3$ denominator 0.1619

                Thus, $n=2$ is optimal so the large core requires 4 units.

            \item We can calculate the speedup using Amdahl's law
                \[ \frac{1}{0.1 + \frac{0.9}{16}} = 6.4\mathrm{x} \]

            \item Yes, with the heterogeneous system we get a speedup of
                8x whereas the homogeneous system achieves only 6x.
        \end{enumerate}
    \item
        \begin{enumerate}[label=\roman*)]
            \item Now the speedup is
                \[ \frac{1}{\frac{0.04}{n} + \frac{0.96}{16-n^2}} \]
                and $n=2$ still maximizes the speedup so we need 4 units for the
                large core.

            \item We can calculate the speedup using Amdahl's law
                \[ \frac{1}{\frac{0.04}{2} + \frac{0.96}{12}} = 10\mathrm{x} \]

            \item We can calculate the speedup using Amdahl's law
                \[ \frac{1}{0.04 + \frac{0.96}{16}} = 10\mathrm{x} \]

            \item No, the homogeneous system is as fast as the heterogeneous
                system and the homogeneous system is easier to design.
        \end{enumerate}
\end{enumerate}

\section{Asymmetric Multicore [400 points]}

\begin{enumerate}[label=\alph*)]
    \item The large core occupies $n^3$ units so the number of small cores is
        $32 - n^3$.

        Using the formula from the lecture notes we get

        \[ Speedup = \frac{1}{\frac{0.2}{n} + \frac{0.8}{32-n^3}} \]

        We can approximate the total execution time as

        \[ t_n = t_{ser} + t_{par} = \frac{0.2}{n} + \frac{0.8}{32-n^3} \]

        and try different values for $n$

        $t_1 = 0.2 + 0.03 = 0.23$

        $t_2 = 0.1 + 0.03 = 0.13$

        $t_3 = 0.07 + 0.16 = 0.23$

        Thus, the optimal configuration is $n=2$ and 24 small cores.

    \item
        \begin{equation*}
            \begin{split}
                E_{ser} &= t_{ser} \times (P_{large\_dynamic} + P_{large\_static}
                + P_{small\_static}) \\
                &= 0.1 \times (6n + n + 0.5(32 - n^3)) \\
                &= 2.6 \, Joules
            \end{split}
        \end{equation*}

        \begin{equation*}
            \begin{split}
                E_{par} &= t_{par} \times (P_{large\_static} + P_{small_dynamic}
                + P_{small_static} \\
                &= 0.03 \times (n + (32 - n^3) + 0.5(32 - n^3)) \\
                &= 1.14 \, Joules
            \end{split}
        \end{equation*}

        \[E_{total} = E_{ser} + E_{par} = 2.6 + 1.14 = 3.74 \, Joules\]

    \item
        \begin{enumerate}[label=\roman*)]
            \item If the large core participate in the computation of the
                parallel portion, $t_{ser}$ stays the same and $t_{par}$
                decreases from $\frac{0.8}{32-n^3}$ to $\frac{0.8}{32-n^3 + n}$
                i.e. a speedup  of $\frac{13}{12}$.

            \item $E_{ser}$ is still 2.6 Joules. For the parallel portion we get
                \begin{equation*}
                    \begin{split}
                        E'_{par} &= t'_{par} \times (P_{large\_static}
                        + P_{large\_dynamic} + P_{small_dynamic}
                        + P_{small_static} \\
                        &= 0.0307 \times (6n + n + (32 - n^3) + 0.5(32 - n^3)) \\
                        &= 1.54 \, Joules
                    \end{split}
                \end{equation*}
                and can calculate
                \[E'_{total} = E_{ser} + E'_{par} = 2.6 + 1.54 = 4.14 \, Joules\]
                The overall increase is $\frac{4.14 \, Joules}{3.74 \, Joules} = 1.12x$.

            \item The performance increases by 1.08x whereas the energy
                consumption increases by 1.12x so the large core collaboration
                is not really an improvement.
        \end{enumerate}

    \item We reduce the size of the large core i.e. $n=1$. Let $s$ be the new
        serial portion and $t_2$ the execution time calculated in (a).
        \begin{equation*}
            \begin{split}
                t_2 &> t'_{ser} + t'_{par} = \frac{s}{n} + \frac{1-s}{32 - n^3} \\
                0.13 &> \frac{s}{1} + \frac{1-s}{31} \\
                0.1 &> s
            \end{split}
        \end{equation*}
        The serial portion should be at most 10\%.

    \item The total power consumption is
        \[ P_{total} = \frac{E_{ser} + E_{par}}{t_{total}} \]
        $E_{par}$ is still $1.14 \, Joules$ whereas
        $E_{ser} = 0.1 \times (D \times n + n + 0.5(32 - n^3) = 0.2D + 1.4 \,
        Joules$.

        Now we have to constrain the total power consumption to
        \begin{equation*}
            \begin{split}
                P_{total} = \frac{0.2D + 1.4 + 1.14}{0.13} &\leq 20 \, Watts \\
                D &\leq 0.3
            \end{split}
        \end{equation*}
        The current dynamic power consumption of the large core is $6n$ Watts
        and we have to decrease it to $0.3n$ Watts. Thus, we have to decrease
        the power consumption by at least 20x.
\end{enumerate}

\section{Runahead Execution [200 points]}

\begin{enumerate}[label=\alph*)]
    \item To get 66 cache hits within 100 load instructions, every cache miss has
        to be followed by 2 cache hits. Thus, the runahead execution has to
        prefetch 2 cache blocks within the memory latency X. To get 2 times to
        the load instruction, the processor has to execute at least $2 \times
        (29 \, ALU + 1 \, BRANCH + 1 \, LD) = 62$ instructions in runahead mode.
        If the processor executes more than 92 instructions, 3 cache blocks
        will be loaded which conflict with each other so the cache hits
        decrease. Therefore, the cache miss latency is $61 < X < 93$.

    \item If the cache miss latency is below 31 cycles i.e. $X < 31$, the
        runahead execution cannot reach the load instruction which always
        incur a cache miss. If the cache miss latency is above 123 i.e. $X
        > 123$ 4 cache blocks are fetched and the first runahead block is
        evicted from cache (all blocks map to same set and regular block is not
        evicted).
    \item If the runahead execution fetches 3 cache blocks, the cache is filled
        in an optimal way because with 4 blocks interference happens and with
        2 blocks the cache is not fully used. Therefore, the minimum number of
        cache misses is 25.
\end{enumerate}

\section{Prefetching [200 points]}

\begin{enumerate}[label=\alph*)]
    \item A cache block holds 4 elements of \verb|a|.

        \textbf{Application A}

        Application A has the following access pattern

        \verb|a[0]|, \verb|a[4]|, \verb|a[8]|, ..., \verb|a[996]|

        and performs $1000/4 = 250$ accesses to memory.

        The first two accesses are misses without a prefetch. Then the
        prefetcher observes a stride of 4 and starts prefetching the correct
        cache blocks until \verb|a[1000]|. All prefetched blocks are used except
        \verb|a[1000]| so the accuracy is $\frac{248}{249}$ and the coverage is
        $\frac{248}{250}$ because the first two misses were not prefetched.

        \textbf{Application B}

        Application B has the following access pattern

        \verb|a[1]|, \verb|a[4]|, \verb|a[16]| \verb|a[64]| \verb|a[256]|

        The prefetcher is not able to find a constant stride across these
        accesses. Hence, the accuracy and coverage are both 0.

    \item
        \textbf{Application A}

        A next-line prefetcher would always prefetch the next cache line so
        \verb|a[4]| would also be prefetched. Thus, accuracy and coverage would
        be both $\frac{249}{250}$.

        \textbf{Application B}

        Application B has an uncommon access pattern that is not supported by
        common prefetchers. Therefore, more sophisticated prefetchers such as
        a Markov prefetcher or runahead execution are required for accurate
        prefetches.

    \item

        \textbf{Application A}

        No, Application A has already a good prefetch metrics with a stride or
        next-line prefetcher so using runahead execution would be an overkill.
        What's more, the processor can enter runahead mode only when it's
        stalled whereas a stride/next-block prefetcher can run in parallel to
        the processor.

        \textbf{Application B}

        Yes, Application B has an uncommon access pattern which could be
        detected using runahead execution.
\end{enumerate}

\section{Cache Coherence [300 points]}

\begin{enumerate}[label=\alph*)]
    \item The cache block \verb|0x511100| in set 1 cannot be in state E in cache
    2 and in state S in cache 3 because the MESI protocol doesn't allow that
      a cache block is in exclusive and shared state at the same time. The
        cosmic rays could have either switched the cache block from I to S in
        cache 3 or from S to E in cache 2.

    \item The first path cannot lead to an incorrect execution because the store
        instruction of processor 3 brings the system back into a consistent
        state (cache 3 becomes M and cache 2 switches to I).

        The second path could lead to an incorrect execution because the first
        instruction could read invalid data if the cosmic ray change was from
        I to S in cache 3 (cache 2 believes it has an exclusive copy and doesn't
        notify the other caches on a write).

    \item
        \begin{enumerate}[label=(\arabic*)]
            \item \verb|st 0x522222C0| Processor 0
            \item \verb|ld 0x5FF00000| Processor 1
            \item \verb|ld 0x5FFFFFC0| Processor 0
            \item \verb|st 0x5FF00000| Processor 3
        \end{enumerate}
\end{enumerate}

\section{Memory Consistency [300 points]}

\begin{enumerate}[label=\alph*)]
    \item Sequential consistency guarantees that each thread executes the
        instructions in the order specified by the programmer. Across threads
        the ordering is ensured by the while-loop and the \verb|flag| variable.
        T1 initializes \verb|Y[0]| to 1 and T0 waits until the flag is set i.e.
        \verb|Y[0]| is initialized. After the flag is set, T0 increments
        \verb|Y[0]| so the final value is 2.

    \item There are several possible instruction interleaving but the final
        value of \verb|b| is either 0 or 1. If T0.1 executes before T1.0 the
        value is 0 or if T1.0 executes before T0.1 the value is 1.

    \item Due to weak consistency the following orderings are possible:

        T1.1 T0.3 T1.0 Final Value 1

        T1.1 T1.0 T0.3 Final Value 2

        T1.0 T1.1 T0.3 Final Value 2

    \item A memory fence between T1.0 and T1.1 ensures the \verb|Y[0]| is
        initialized before the flag is set.

        A memory fence between T0.2 and T0.3 ensures that \verb|Y[0]| is not
        incremented before the flag is set from T1.
\end{enumerate}

\section{BONUS: Building Multicore Processors [250 points]}

\begin{enumerate}[label=\alph*)]
    \item The loop body is not parallelizable because \verb|past| and
    \verb|current| are stored in registers on the same core. Hence, we can
        ignore the cycle information. The 13th loop iteration is dependent on
        the 1st iteration so we can execute 12 consecutive iterations in
        parallel i.e. the serial portion of the program is $\frac{1}{12}$. Using
        Amdahl's law with an infinite number of cores we get a maximum speedup
        of 12x.

    \item 12 cores i.e. one core per parallel iteration.

    \item Assign iteration $i$ to processor $i \, \% \, 12$.

    \item No. On a single core machine one memory access is performed every
        4 cycles so 12 iterations take 900 cycles. To achieve a 12x speedup,
        each core needs to complete it's work in 75 cycles which is impossible
        as a memory access alone requires 100 cycles.

    \item The maximum speedup is achieved if every cache hit is used as best as
        possible. Thus, 3 cores are required to achieve almost a 3x speedup.

    \item Core 1 runs iterations i+0, i+1, i+2, i+3

        Core 2 runs iterations i+4, i+5, i+6, i+7

        Core 3 runs iterations i+8, i+9, i+10, i+11

\end{enumerate}

\end{document}


\documentclass[a4paper]{article}
\usepackage[utf8]{inputenc}
\usepackage[a4paper, margin=1in]{geometry}
\usepackage{enumitem}
\usepackage{hyperref}
\usepackage{listings}
\usepackage{tabularx}
\usepackage{xcolor}
\usepackage{amsmath}
\usepackage{mathtools}
\DeclarePairedDelimiter\ceil{\lceil}{\rceil}
\DeclarePairedDelimiter\floor{\lfloor}{\rfloor}

\renewcommand{\arraystretch}{1.5}

\definecolor{codegreen}{rgb}{0,0.6,0}
\definecolor{codegray}{rgb}{0.5,0.5,0.5}
\definecolor{codepurple}{rgb}{0.58,0,0.82}
\definecolor{backcolour}{rgb}{0.95,0.95,0.92}

\lstdefinestyle{mystyle}{
    backgroundcolor=\color{backcolour},   
    commentstyle=\color{codegreen},
    keywordstyle=\color{magenta},
    numberstyle=\tiny\color{codegray},
    stringstyle=\color{codepurple},
    basicstyle=\ttfamily\footnotesize,
    breakatwhitespace=false,         
    breaklines=true,                 
    captionpos=b,                    
    keepspaces=true,                 
    numbers=left,                    
    numbersep=5pt,                  
    showspaces=false,                
    showstringspaces=false,
    showtabs=false,                  
    tabsize=2
}
\lstset{style=mystyle}

\title{Computer Architecture HW1}
\author{Fabian Wüthrich}

\begin{document}

\maketitle

\begin{enumerate}
    \item Critical Paper Reviews (1000 points)

        see \href{https://safari.ethz.ch/review/architecture20/}{here}
    
    \item Main Memory (150 points)

    \begin{enumerate}
        \item The memory access pattern of application A could induce more \textit{row
            buffer misses} whereas application B has more \textit{row buffer hits}. As a
            \textit{row buffer miss} has a higher latency as a \textit{row buffer hit}, 
            a memory request of application A takes longer.

        \item A \textit{row buffer miss} consumes more energy as a \textit{row buffer hit}
            (e.g. precharge of bitlines necessary). As application A has more \textit{row
            buffer misses} it will also use a larger amount of memory energy.

        \item The scheduling algorithm in the memory controller is most likely FR-FCFS. This
            algorithm could prioritize application B (\textit{memory performance hog}) because it has 
            a better row buffer locality degrading the performance of application A.
        
        \item Accessing and refreshing a row in DRAM is actually the same process so this policy prevents
            unnecessary refreshes while still maintaining a retention time of 64ms which is good for
            performance and energy consumption. The downside of this policy is that additional circuitry
            is required to keep track of the last access time.

        \item Application B seems to open many different so the new policy will not refresh these rows
            again. Application A has a good row buffer locality and doesn't open many rows which increases
            refresh energy consumption. It is important to note that the total memory energy consumption of
            application B is still higher because of the many row buffer misses.
    \end{enumerate}

    \item DRAM Refresh - Utilization (150 points)
        \begin{enumerate}
            \item We have 4 channels$\times$2 ranks$\times$8 banks$\times$32K rows = $2^{21}$ rows in total
                and $\frac{1.024s}{64ms} = 2^4$ refreshes per row. Thus, $2^{25}$ refreshes across all 
                four channels.

            \item The command bus of each memory channel is occupied for $2^{23}\cdot5ns$ by refreshes. For a
                time-span of $1.024s$ this gives an utilization of $\frac{2^{23}\cdot5ns}{1.024s}=4.096\%$ 

            \item A refresh uses only the command bus so the data bus utilization is 0.

            \item Each bank issues $2^{15}$ banks $\times$ $2^4$ refreshes/bank = $2^{19}$ refreshes and each
                refresh takes $40ns$. Therefore, we have an utilization of
                $\frac{2^{19}\cdot40ns}{1.024s}=2.048\%$
            
            \item We have $2^5$ rows that are refreshed 16 times, $2^9$ rows get refreshed 8 times and
                the remaining $2^{21} - 2^5 - 2^9$ rows are only 4 times refreshed. Therefore,
                \[ 2^5\cdot16 + 2^9\cdot8 + (2^{21} - 2^5 - 2^9)\cdot4 = 8391040\] refreshes in total.

                Then we get an utilization of $\frac{8391040\cdot5ns}{4*1.024s}=1.0243\%$ over all four 
                command buses.
            
            \item The new distribution still uses only the command bus so the data bus utilization is zero. 

            \item We have 64 banks so we get a bank utilization of $\frac{8391040\cdot40ns}{64\cdot1.024}=
                0.5121\%$

            \item The system has to track three different states so we use 2 bits per row. To track the state
                of all rows we need $2$ bits $\times2^{21}$ rows $=4$ Mbit

            \item When using bloom filters the number of refreshes per interval stays the same but we have
                to add the false-positives to the number of rows which are refreshed in each interval. For
                the $64ms$ interval the bloom filter classifies $2^5$ rows correctly plus 
                $(2^{21}-2^5)\cdot2^{-20}$ false-positives so $n_{64}=2^5 + (2^{21}-2^5)\cdot2^{-20}$ rows are 
                refreshed. With a similar reasoning we get $n_{128}=2^9 + (2^{21}-2^9-2^5)\cdot2^{-8}$ and
                $n_{256} = 2^{21} - n_{64} - n_{128}$. Therefore, we get $16 \cdot n_{64} + 8\cdot n_{128} +
                4\cdot n_{256} = 8423823$ refreshes.

                Similar to (e) we get a command bus utilization of $\frac{8423823\cdot5ns}{4*1.024s}=1.028\%$

                Still no data bus used so zero data bus utilization.

                Similar to (g) we get a bank utilization of $\frac{8423823\cdot40ns}{64\cdot1.024}=
                0.5141\%$
        \end{enumerate}
    \item RowHammer (150 points)
        \begin{enumerate}
            \item The given code doesn't work because the \verb|STORE| instruction is cached in
                the row buffer even thought the other caches are circumvented using \verb|CLFLUSH|.

            \item We assume that the MSB of the physical address is use to identify the channel. Then
                we use the next two bits to address the bank and the next six bits are use to identify
                the row. The remaining seven LSBs are used to address the individual bytes.

                Using this address scheme only \textbf{Code 2a} can induce RowHammer errors because both 
                addresses map to the same bank on the same channel (three MSBs are equal). RowHammer 
                requires two aggressor rows on the same bank to circumvent the row buffer.
        \end{enumerate}
    \item RowHammer Mitigation Mechanisms
        \begin{enumerate}
            \item We need 18 bits to store $T$ for each row. Then the total number of bits required
                is $2 \cdot 8 \cdot 2^{15} \cdot 18 \mathrm{bits} = 9 \mathrm{Mbits}$.
            
            \item Now we have 16 banks and $2^{16}$ rows per bank so we get
                $2 \cdot 16 \cdot 2^{16} \cdot 18 \mathrm{bits} = 36 \mathrm{Mbits}$.
            
            \item To analyze mechanism A we look at a $17 \times 64ms$ interval. In this interval
                row \verb|0x5FF02| had 170 K activations so with T = 164 K mechanism A issued one
                additional activation. Thus, including the other rows with a higher activation count
                we get
                \begin{equation*}
                    \floor*{\frac{17 \cdot 10K}{T}} + \floor*{\frac{17 \cdot 32K}{T}} + 
                    \floor*{\frac{17 \cdot 64K}{T}} + \floor*{\frac{17 \cdot 73K}{T}} = 17
                \end{equation*}
                additional activations in $17 \times 64ms$. Note that we ignored the remaining
                $17 \times 301$ activations because they are distributed across the rows such
                that $T$ does not exceed for a specific row. In summary, mechanism A produces
                one additional activation within a 64ms time interval.

                For mechanism B we model a random variable $X$ as the number of additional 
                activations in 64ms. $X$ has a binomial distribution with parameters $n=480K$
                and $p=0.001$. Therefore, we get $\mathbf{E}[X] = 480K \cdot 0.001 = 480$ 
                additional activations within a 64ms time interval.

            \item As the memory access pattern doesn't change the answer is the same as before.

            \item For both techniques the memory controller must know which rows are adjacent to
                each other. It is a challenge to find two adjacent rows because the memory controller
                cannot rely on the physical address as they are often remapped within the DRAM chip
                and manufacturers don't disclose the remapping.
            
            \item To address this challenge we need intelligent memory controllers and a better
                interface between the memory controller and DRAM chips.
        \end{enumerate}
    \item VRL: Variable Refresh Latency (150 points)
        \begin{enumerate}
            \item With VRL applied a full refresh has latency $T$ whereas a partial
                refresh has latency $0.6T$. 10\% of the rows can tolerate one 
                partial refresh followed by a full refresh so the refresh latency
                for these rows is $0.1 \times (0.5 \times 0.6T + 0.5 \times T)=0.08T$.
                Similarly, we get $0.6 \times (0.75 \times 0.6T + 0.25 \times T)=0.42T$
                and $0.3 \times (0.875 \times 0.6T + 0.125 \times T)=0.195T$ as new latency
                for the other two groups. Overall, the new refresh latency of a bank
                is \[0.695T\]
            
            \item In a $128ms$ interval 90\% of the rows are refreshed once and 10\% of
                the rows are refreshed twice. A bank is busy for 
                $0.9N \times \frac{0.2}{N} + 2 \times 0.1N \times \frac{0.2}{N} = 0.22ms$
                during a refresh. Therefore, we have a refresh overhead of 
                \[\frac{0.22}{128}\]

            \item With VRL a partial refresh costs $0.12/Nms$ whereas a full
                refresh still requires $0.2/Nms$. The refresh pattern for every cell is
                repeated every $4 \times 128ms$. In this timespan 5\% of the rows have
                1 full refresh and 3 partial refreshes. 65\% of the rows have 2 full
                refreshes and 2 partial refreshes. 10\% of the rows have 4 full refreshes
                and 4 partial refreshes. 20\% of the rows have 4 full refreshes. The total
                time spent for refresh is then
                \begin{align*} 
                    0.2 \ N \times (0.05N \times 1 + 0.65N \times 2 + 0.1N \times 4 + 0.2N \times 4) + \\
                    0.12 \ N \times (0.05N \times 3 + 0.65N \times 2 + 0.1N \times 4) &= 0.732ms
                \end{align*}
                Therefore, we have a refresh overhead of
                \[\frac{0.732}{512}\] with a reduction of $\approx 16.8\%$ compared to the baseline.
        \end{enumerate}
    \item BONUS: DRAM Refresh - Energy (150 points)
        \begin{enumerate}
            \item We have a total capacity of $2^{60}$ bytes and each row has a size of $2^{13}$ bytes.
                Hence, the memory system has $2^{47}$ rows.

            \item The memory system has to guarantee a minimum retention time of $64ms$ so it has to refresh
                all rows within $64ms$. The memory controller issues a REF command every $7.8\mu s$ so 
                $\frac{64ms}{7.8\mu s}=8205$ REF commands reach the DRAM chip in $64ms$. Thus, each REF
                command must refresh $2^{47}/8205$ rows.

            \item To calculate the refresh power we use Eq. 18 from the Micron Technical Note.
                \[P_{ds}(REF) = (IDD5 - IDD3N) \times V_{DD}\]
                From the datasheet we get $V_{DD}=1.5V$ and if we assume DDR3-1866x16 we get $IDD5=175mA$
                and $IDD3N=55mA$ from Table 19. Therefore,
                \[P_{ds}(REF) = (175mA - 55mA) \times 1.5V = 180mW\]

            \item For a refresh cycle ($64ms$) we have $E_{ref} = 180mW \cdot 64ms = 11.52 mJ$. 

                For a day we get $E_{day} = 180mW \cdot 86400 s  = 15552 J$
        \end{enumerate}
\end{enumerate}

\end{document}

